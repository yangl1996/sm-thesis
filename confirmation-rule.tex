\chapter{Confirmation Rule}
\label{apx:confirmation}

In this appendix we give the detailed calculation of the confidence intervals of the votes a proposer block receives. It is used when confirming a leader proposer block, as mentioned in \S\ref{sec:confirmation}.

Consider the scenario where there are $n$ proposer blocks at level $l$, and let $\mathcal P =\{B^P_1, B^P_2, \ldots, B^P_n\}$ denote the set of proposer blocks at level $l$. Now we want to count the number of votes each block will get with confidence $1-\epsilon$.

Suppose $B^P_i$ gets $v_i$ votes. Here a vote stands for a voter block which is on the longest chain of its voter tree and votes for $B^P_i$. 
Let $\mathcal V_i =\{B^V_{i_1}, B^V_{i_2}, \ldots, B^V_{i_{v_i}}\}$ denote the set of votes that $B^P_i$ has. For every vote $B^V_{i_j}$, let $d_{i_j}$ denote its depth, which is the number of blocks appended to voter block $B^V_{i_j}$ in the longest chain, plus one.

Now, for each vote $B^V_{i_j}$ with depth $d_{i_j}$, we want to calculate the probability $P_{i_j}$ of it being permanent. 
% \textit{not} being overtaken by a hypothetical, unreleased adversarial chain any time in the future. 
To do so, we consider a potential private double-spend attack, assuming an adversarial party is trying to overturn the voting results to elect a different proposer block $B^P_A$ as the leader block of level $l$. 
Note that $B^P_A$ could either be a block in $\mathcal P$, i.e. publicly known, or a block the adversary has privately mined but not released. 
To elect $B^P_A$ as the leader block of level $l$, the adversarial party would need to mine its own voter chains to overturn some existing votes to vote for $B^P_A$. 

We want to compute the probability of this happening.
% we need to reason about the blocks the adversary has already mined as well as those it will mine in the future. 
However, we do not know when the adversary started mining voter blocks for $B^P_A$.
% , which means we do not know how long the adversary could have been mining its private blocks. 
% However, from the visible state of the blockchain, we can estimate when $B^P_i$ was mined; assuming $B^P_A$ was mined at least as late as $B^P_i$, this gives us a bound. 
Notice that the adversary has no incentive to mine voter blocks for $B^P_A$ until $B^P_i$ has been mined and released. 
Since the honest nodes are always releasing blocks, we can use the average depth of the votes for $B^P_i$ in the public voter trees to estimate the time passed since $B^P_i$ was released, hence  bounding the expected number of votes the adversary could have accumulated on their private fork in the same amount of time. 
That is, since block inter-arrivals are exponentially distributed, the number of blocks mined since block $B^P_i$ was proposed is a Poisson random variable, with rate equal to its mean. 
This quantity can be related to the time elapsed since $B^P_i$ was released via the block mining rate.
% To do this, we estimating how deep those private voter chains are on average. 

More precisely, as an honest node, we assume the fraction of adversarial hashing power is $\beta$, and we can empirically estimate the average depth of existing public votes as $\bar{d}=\sum_{ij}{d_{i_j}}/\sum_{i}{v_i}$ and the forking rate $\alpha$~\footnote{The fraction of blocks not on the longest chain out of all blocks.} of public voter chains. 
Since there are many voter chains, these estimates converge quickly to their true means. 
Then, we calculate the estimated average depth of a \emph{private} voter chain, denoted as $\bar{d_A}$, to be
% \gf{Yes, the sentences starting with "we observe that" to the end of this para  (through the following equations) need a good bit more explanation. Minor nits: In the defn of $\bar{d}$, $v_i$ was earlier defined as the total number of votes, not the honest number  of votes,
% but $\bar{d}$ is defined as the average depth of honest votes. How are you defining an "honest vote" here? 
% Also, the sentence starting with "We observe" doesn't parse.}
% \ly{``honest'' is better defined as ``public'' here, and I have changed the definitions correspondingly. Also, I added the sentence below the next equation. Does this sound like enough explanantion?}
$$\bar{d_A} = \frac{\beta \bar{d}}{(1-\alpha)(1-\beta)}.$$
Here the $1/(1-\alpha)$ term accounts for forking in public voter chains and assumes that the malicious private voter chains do not fork. The $\beta / (1-\beta)$ term accounts for the ratio of hashing power between the honest users ($1-\beta$) and the malicious users ($\beta$).
This expected depth $\bar{d_A}$ can be used as an estimate of the rate of the Poisson random variable of the number of blocks in the adversary's  private chain. 


Since each voter chain follows the longest-chain rule, the calculation for $P_{i_j}$ is the same as in \bitcoin
$$P_{i_j} = F_{\text{Pois}}(d_{i_j};\bar{d_A}) - \sum_{k=0}^{d_{i_j}}f_{\text{Pois}}(k;\bar{d_A})\frac{\beta}{1-\beta}^{d_{i_j} + 1 - k}.$$
Here $F_{\text{Pois}}(x;\lambda)$ is the cumulative distribution function and $f_{\text{Pois}}(x;\lambda)$ is the probability mass function of Poisson distribution with rate parameter $\lambda$.
In this expression, the first term is the probability that the adversary has mined fewer than $d_{i_j}+1$ blocks, in which case it cannot currently overtake the main chain. 
The second term computes, for each possible length of the adversary's chain, the probability that the adversary overtakes the public voter chain in the future by mining faster. 

Given $P_{i_j}$, we can now calculate the confidence interval of votes on each proposer block. For proposer block $B_i^P$ and each of its votes $B^V_{i_j}$, let $\tilde V_{i_j}$ be the random variable where
$$\tilde V_{i_j} = \begin{cases}
    1, & \text{if vote } B^V_{i_j} \text{ is secure forever (permanent)} \\
    0, & \text{if vote } B^V_{i_j} \text{ will be overturned}
  \end{cases}.$$
With some abuse of notation, let $v_i$ be the random variable equal to the number of secure votes of $B^P_i$. We have
$$v_i = \sum_{j}B^V_{i_j}.$$
Note that $\tilde V_{i_j}\sim \text{Bernoulli}(P_{i_j})$. Then the lower confidence bound of votes on $B_i^P$ (denoted as $\left \lfloor v_{i} \right \rfloor$) can be obtained by calculating in $\epsilon$-quantile of random variable $v_i$.

In real-world implementations, given the complexity of such computation, its closed-form approximation may be used. We can approximate $v_i$ using a Gaussian distribution $\mathcal N(\mu_i, \sigma_i^2)$ where
$$\mu_i=\sum_j P_{i_j}.$$
$$\sigma_i^2 = \sum_j P_{i_j}(1-P_{i_j}).$$
Using the closed-form approximation of the quantile function of normal distribution, we have
$$\left \lfloor v_{i} \right \rfloor \approx \mu_i - \sigma_i \sqrt{\ln{
\frac{1}{\epsilon^2}} - \ln{\ln{\frac{1}{\epsilon^2}}} - \ln{(2\pi)}}.$$

Now, we consider the upper confidence bound of votes on $B_i^P$ (denoted as $\left \lceil v_{i} \right \rceil$). Here, we want to defend against the worst case where for each $B_i^P$, only $\left \lfloor v_{i} \right \rfloor$ votes are retained, and the adversarial party controls the remaining votes (we let $\left \lceil v_A \right \rceil$ denote the number of such votes). Recall that each voter chain can only vote for each proposer level once. For a system with $m$ voter chains, we have
$$\left \lceil v_A \right \rceil = m - \sum_{i}\left \lfloor v_{i} \right \rfloor.$$
The adversarial party will use those votes to vote for $B^P_A$. Since $B^P_A$ could be any block in $\mathcal P$, we have
$$\left \lceil v_i \right \rceil = \left \lfloor v_{i} \right \rfloor + \left \lceil v_A \right \rceil.$$
$B^P_A$ could also be a block which the adversarial party mines but has not released. In such case, the upper bound of votes on $B^P_A$ is just $\left \lceil v_A \right \rceil$. Finally, to select the leader of level $l$, we search for the block $B^P_L \in \mathcal P$ satisfying $\left \lfloor v_{L} \right \rfloor > \left \lceil v_i \right \rceil$ for every $i \neq L$ and $\left \lfloor v_{L} \right \rfloor > \left \lceil v_A \right \rceil$. In words, we select the block whose lower bound of votes is higher than the upper bound of any other known or unknown proposer block in the same level.

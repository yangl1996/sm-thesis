% -*-latex-*-
% 
% For questions, comments, concerns or complaints:
% thesis@mit.edu
% 
%
% $Log: cover.tex,v $
% Revision 1.8  2008/05/13 15:02:15  jdreed
% Degree month is June, not May.  Added note about prevdegrees.
% Arthur Smith's title updated
%
% Revision 1.7  2001/02/08 18:53:16  boojum
% changed some \newpages to \cleardoublepages
%
% Revision 1.6  1999/10/21 14:49:31  boojum
% changed comment referring to documentstyle
%
% Revision 1.5  1999/10/21 14:39:04  boojum
% *** empty log message ***
%
% Revision 1.4  1997/04/18  17:54:10  othomas
% added page numbers on abstract and cover, and made 1 abstract
% page the default rather than 2.  (anne hunter tells me this
% is the new institute standard.)
%
% Revision 1.4  1997/04/18  17:54:10  othomas
% added page numbers on abstract and cover, and made 1 abstract
% page the default rather than 2.  (anne hunter tells me this
% is the new institute standard.)
%
% Revision 1.3  93/05/17  17:06:29  starflt
% Added acknowledgements section (suggested by tompalka)
% 
% Revision 1.2  92/04/22  13:13:13  epeisach
% Fixes for 1991 course 6 requirements
% Phrase "and to grant others the right to do so" has been added to 
% permission clause
% Second copy of abstract is not counted as separate pages so numbering works
% out
% 
% Revision 1.1  92/04/22  13:08:20  epeisach

% NOTE:
% These templates make an effort to conform to the MIT Thesis specifications,
% however the specifications can change.  We recommend that you verify the
% layout of your title page with your thesis advisor and/or the MIT 
% Libraries before printing your final copy.
\title{Design and Implementation of a High Performance Blockchain System}

\author{Lei Yang}
% If you wish to list your previous degrees on the cover page, use the 
% previous degrees command:
       \prevdegrees{B.S., Peking University (2018)}
% You can use the \\ command to list multiple previous degrees
%       \prevdegrees{B.S., University of California (1978) \\
%                    S.M., Massachusetts Institute of Technology (1981)}
\department{Department of Electrical Engineering and Computer Science}

% If the thesis is for two degrees simultaneously, list them both
% separated by \and like this:
% \degree{Doctor of Philosophy \and Master of Science}
\degree{Master of Science in Computer Science and Engineering}

% As of the 2007-08 academic year, valid degree months are September, 
% February, or June.  The default is June.
\degreemonth{May}
\degreeyear{2020}
\thesisdate{May 15, 2020}

%% By default, the thesis will be copyrighted to MIT.  If you need to copyright
%% the thesis to yourself, just specify the `vi' documentclass option.  If for
%% some reason you want to exactly specify the copyright notice text, you can
%% use the \copyrightnoticetext command.  
%\copyrightnoticetext{\copyright IBM, 1990.  Do not open till Xmas.}

% If there is more than one supervisor, use the \supervisor command
% once for each.
\supervisor{Mohammad Alizadeh}{Associate Professor of Electrical Engineering and Computer Science}

% This is the department committee chairman, not the thesis committee
% chairman.  You should replace this with your Department's Committee
% Chairman.
\chairman{Leslie A. Kolodziejski}{Professor of Electrical Engineering and Computer Science\\Chair, Department Committee on Graduate Students}

% Make the titlepage based on the above information.  If you need
% something special and can't use the standard form, you can specify
% the exact text of the titlepage yourself.  Put it in a titlepage
% environment and leave blank lines where you want vertical space.
% The spaces will be adjusted to fill the entire page.  The dotted
% lines for the signatures are made with the \signature command.
\maketitle

% The abstractpage environment sets up everything on the page except
% the text itself.  The title and other header material are put at the
% top of the page, and the supervisors are listed at the bottom.  A
% new page is begun both before and after.  Of course, an abstract may
% be more than one page itself.  If you need more control over the
% format of the page, you can use the abstract environment, which puts
% the word "Abstract" at the beginning and single spaces its text.

%% You can either \input (*not* \include) your abstract file, or you can put
%% the text of the abstract directly between the \begin{abstractpage} and
%% \end{abstractpage} commands.

% First copy: start a new page, and save the page number.
\cleardoublepage
% Uncomment the next line if you do NOT want a page number on your
% abstract and acknowledgments pages.
% \pagestyle{empty}
\setcounter{savepage}{\thepage}
\begin{abstractpage}
% $Log: abstract.tex,v $
% Revision 1.1  93/05/14  14:56:25  starflt
% Initial revision
% 
% Revision 1.1  90/05/04  10:41:01  lwvanels
% Initial revision
% 
%
%% The text of your abstract and nothing else (other than comments) goes here.
%% It will be single-spaced and the rest of the text that is supposed to go on
%% the abstract page will be generated by the abstractpage environment.  This
%% file should be \input (not \include 'd) from cover.tex.
\bitcoin is the first fully-decentralized permissionless blockchain protocol to achieve a high level of security: the ledger it maintains has guaranteed liveness and consistency properties as long as the adversary has less compute power than the honest nodes. However, its throughput is only $7$ transactions per second and the confirmation latency can be up to hours. \prism is a new blockchain protocol that is designed to achieve a natural scaling of \bitcoin's performance while maintaining its full security guarantees. We present an implementation of \prism that achieves a throughput of over $70,000$ transactions per second and confirmation latency of tens of seconds on networks of up to $1000$ EC2 Virtual Machines. The code can be found at \cite{prismcode}.

\end{abstractpage}

% Additional copy: start a new page, and reset the page number.  This way,
% the second copy of the abstract is not counted as separate pages.
% Uncomment the next 6 lines if you need two copies of the abstract
% page.
% \setcounter{page}{\thesavepage}
% \begin{abstractpage}
% % $Log: abstract.tex,v $
% Revision 1.1  93/05/14  14:56:25  starflt
% Initial revision
% 
% Revision 1.1  90/05/04  10:41:01  lwvanels
% Initial revision
% 
%
%% The text of your abstract and nothing else (other than comments) goes here.
%% It will be single-spaced and the rest of the text that is supposed to go on
%% the abstract page will be generated by the abstractpage environment.  This
%% file should be \input (not \include 'd) from cover.tex.
\bitcoin is the first fully-decentralized permissionless blockchain protocol to achieve a high level of security: the ledger it maintains has guaranteed liveness and consistency properties as long as the adversary has less compute power than the honest nodes. However, its throughput is only $7$ transactions per second and the confirmation latency can be up to hours. \prism is a new blockchain protocol that is designed to achieve a natural scaling of \bitcoin's performance while maintaining its full security guarantees. We present an implementation of \prism that achieves a throughput of over $70,000$ transactions per second and confirmation latency of tens of seconds on networks of up to $1000$ EC2 Virtual Machines. The code can be found at \cite{prismcode}.

% \end{abstractpage}

\cleardoublepage

\section*{Acknowledgments}

This research was performed under the supervision of my advisor Professor Mohammad Alizadeh and in collaboration with Vivek Bagaria, Gerui Wang, Professor David Tse, Professor Giulia Fanti, and Professor Pramod Viswanath. I would like to thank them for the help, guidance, and support along the way. It would be impossible to finish this project without these amazing collaborators.

A simple thanks can never express my gratitude towards Mohammad, for being the best advisor and friend one could ever imagine. He provides me with unreserved support and wisdom when I need them the most, and shows me the elegance and joy of system research. I will be forever grateful for the time and energy he pours into me and the project. I feel extremely lucky to be able to work with Mohammad for the past two years, and can not wait for the endeavors with him in the upcoming years.

I would also like to thank Professor David Tse for all the guidance and advice. Discussions with him have always been fruitful. I can never learn enough from his passion for perfection and crave for simple yet powerful mathematical theories. My thanks also go to Professor Hari Balakrishnan, Professor Pramod Viswanath, Professor Giulia Fanti, and Professor Sreeram Kannan for all the help, discussions, and advice. I am lucky to be able to learn from them all.

I would like to thank my labmates and friends, especially Hongzi Mao, Vivek Bagaria, Prateesh Goyal, Lijie Fan, Venkat Arun, InHo Cho, Arjun Balasingam, Vibhaalakshmi Sivaraman, Mehrdad Khani, Parimarjan Negi, Seo Jin Park, and Zeyuan Shang, for the joy, advice, and support they provide. I would like to thank Sheila Marian for her help as I navigate around MIT. I'm also thankful to MIT and CSAIL for providing an amazing environment for me to pursue my passions.

Lastly but most importantly, I thank my girlfriend Xinyue for her unconditioned support, company, and love. Being with her makes me feel courageous and excited about this journey. Finally, I turn to my parents, Ya'nan and Fan. Words can not express my gratitude here. Thank you for raising me, making me who I am, and always being there for me; this thesis is dedicated to you.


%%%%%%%%%%%%%%%%%%%%%%%%%%%%%%%%%%%%%%%%%%%%%%%%%%%%%%%%%%%%%%%%%%%%%%
% -*-latex-*-

\tikzstyle{ledger} = [draw, fill=blue!20, rectangle, 
    minimum height=4em, minimum width=6em, text centered, text width=5em]
\tikzstyle{blockchain} = [draw, fill=red!20, rectangle, minimum height=4em, minimum width=6em, text centered, text width=5em]
\tikzstyle{miner} = [draw, fill=green!20, rectangle, minimum height=4em, minimum width=6em, text centered, text width=5em]

\tikzstyle{database} = [draw, fill=yellow!20, rectangle, rounded corners, text width=5em, minimum height=3em, minimum width=6em, text centered]


% The block diagram code is probably more verbose than necessary
\begin{tikzpicture}[auto, node distance=2.8cm,>=latex']
    \node [database] (blockchaindb) {Block Structure Database};
    \node [blockchain, above=0.5cm of blockchaindb] (blockchain) {Block Structure Manager};
    \node [ledger, left of=blockchaindb] (ledger) {Ledger Manager};
    \node [miner, right of=blockchaindb] (miner) {Miner};
    \node [database, left of=ledger] (utxodb) {UTXO Database};
    \node [database, right of=blockchain] (mempool) {Memory Pool};
    \node [database, left of=blockchain] (blockdb) {Block Database};
    \node [above=0.5cm of blockchain] (peers) {Peers};
    \node [right of=peers] (newtx) {New Transactions};
    \draw [<->] (blockchaindb) -- node[name=a] {} (blockchain);
    \draw [->] (blockchaindb) -- node[name=b] {} (miner);
    \draw [<->] (blockchaindb) -- node[name=c] {} (ledger);
    \draw [<-] (blockchain) -- node[name=d] {} (miner);
    \draw [<->] (miner) -- node[name=e]{} (mempool);
    \draw [->] (blockchain) -- node[name=f]{} (mempool);
    \draw [<->] (blockchain) -- node[name=g]{} (blockdb);
    \draw [<->] (ledger) -- node[name=h]{} (utxodb);
    \draw [<-] (ledger) -- node[name=i]{} (blockdb);
    \draw [<->] (peers) -- node[name=j]{} (blockchain);
    \draw [->] (newtx) -- node[name=k]{} (mempool);


\end{tikzpicture}